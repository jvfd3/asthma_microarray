% !TeX root = ./LaTeX/asthma_microarray.tex

% This is samplepaper.tex, a sample chapter demonstrating the
% LLNCS macro package for Springer Computer Science proceedings;
% Version 2.21 of 2022/01/12
%
\documentclass[runningheads,orivec]{llncs}
\usepackage{multirow} % Span rows
\usepackage{amsmath} % Simbologia matemática
\usepackage{amssymb} % Simbologia Matemática
\usepackage{xcolor} % color rows
\usepackage{colortbl} % color rows
\usepackage{hyperref} % Links
\usepackage{url} % Links
\definecolor{lightgray}{HTML}{f3f3f3} % color rows
\usepackage[T1]{fontenc}
% T1 fonts will be used to generate the final print and online PDFs,
% so please use T1 fonts in your manuscript whenever possible.
% Other font encondings may result in incorrect characters.
%
\usepackage{graphicx}
% Used for displaying a sample figure. If possible, figure files should
% be included in EPS format.
%
% If you use the hyperref package, please uncomment the following two lines
% to display URLs in blue roman font according to Springer's eBook style:
%\usepackage{color}
%\renewcommand\UrlFont{\color{blue}\rmfamily}
%\urlstyle{rm}
%
%
\title{Mapping Relevant Genes in Severe Asthma: A Computational Strategy for Biomarker Discovery}
%
\titlerunning{Mapping Relevant Genes in Severe Asthma}
% If the paper title is too long for the running head, you can set
% an abbreviated paper title here
%

\author{\
  Milenna Machado Pirovani \inst{1, 2} \orcidID{0000-0002-5060-9418} \and
  Leonardo Henrique da Silva \inst{1} \orcidID{0000-0003-2341-5906} \and
  João Vítor Fernandes Dias \inst{1} \orcidID{0000-0002-8156-9551} \and
  Luana Bastos \inst{1} \and
  Letícia Gontijo \inst{1} \and
  Rafael Lemos \inst{1} \and
  Diego Mariano \inst{1} \orcidID{0000-0002-5899-2052} \and
  Marcos Augusto dos Santos \inst{1, 3} \and
  Raquel C. de Melo-Minardi \inst{1, 3} \orcidID{0000-0001-5190-100X} \ % \and
  % Second Author\inst{2,3}\orcidID{1111-2222-3333-4444} \and
}
%
\authorrunning{Pirovani, et al.}
% First names are abbreviated in the running head.
% If there are more than two authors, 'et al.' is used.
%
\institute{\
  Universidade Federal de Minas Gerais (UFMG), Belo Horizonte, Brazil \and
  \email{milennapirovani@hotmail.com} \and
  Same contribution level. \
  % ABC Institute, Rupert-Karls-University Heidelberg, Heidelberg, Germany\\
  % \email{\{abc,lncs\}@uni-heidelberg.de}
}
