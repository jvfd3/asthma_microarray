\begin{abstract}
  Asthma is a clinically heterogeneous, chronic inflammatory disorder of the airways, characterized by phenotypic variability that hampers accurate diagnosis and the development of tailored treatment strategies. The identification of robust molecular biomarkers is thus crucial for supporting disease stratification and informed therapeutic decision-making. In this study, we aimed to identify genes with high discriminatory power between patients with severe asthma and healthy controls by applying supervised statistical learning to publicly available transcriptomic data. We analyzed the gene expression dataset of severe asthmatic patients from the NCBI GEO repository using logistic regression for feature selection and classification. Our results suggest sets of genes potentially related to asthma, underscoring the importance of non-coding RNAs and pseudogenes. These results refine and complement previously published findings. Additionally, they highlight the value of reanalyzing public omics data with statistical models to generate novel hypotheses on disease mechanisms.

  \keywords{Asthma \and Biomarkers \and Bioinformatics.}
\end{abstract}